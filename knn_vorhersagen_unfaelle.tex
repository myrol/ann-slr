\documentclass{scrartcl}
\usepackage[ngerman]{babel} % default Sprache für alles (Datum usw.)
\usepackage[affil-it]{authblk}
\usepackage{setspace} % Zeilenabstand
\usepackage{csquotes} % um ,,hello" zu benutzen
\usepackage[sorting=none]{biblatex} 
\usepackage[most]{tcolorbox} % Codeblocks
\usepackage{graphicx}
\DefineBibliographyStrings{ngerman}{ % u.a. -> et al.
   andothers = {et al\adddot}
}
\definecolor{gray}{rgb}{0.94, 0.94, 0.95}
\setlength{\parindent}{0pt} % Sets the paragraph indent to 0

\addbibresource{references.bib}

\begin{document}

\begin{titlepage}

   \subject{Literatur-Seminar-Arbeit}
   \title{Vorhersagen von Verkehrsunfällen mithilfe künstlicher neuronaler Netze}
   \author{Erik Rohr}
   \affil{Fachbereich Informatik (02) - Hochschule Bonn-Rhein-Sieg}
   \publishers{\parbox[b][12cm]{\textwidth}{\centering Betreuerin: Doerthe Vieten}}
   \date{\today}

   \maketitle

\end{titlepage}

\newpage
\onehalfspacing

\section*{Abstrakt}
$\ll$ Kurz beschreiben $\gg$

\newpage
\tableofcontents
\newpage

\section{Einleitung}

Jedes Jahr sterben weltweit ca. 1,19 Millionen Menschen in Verkehrsunfällen
(RTA, engl.: \enquote{road traffic accident}) \cite{who}.
Es ist unabdingbar, dass der Straßenbau die Sicherheit der Straßenverkehrsteilnehmer
priorisiert und das Risiko auf Verkehrsunfälle minimiert.
\medskip \\
Es gibt eine Vielzahl an Faktoren, die das Risiko auf RTA erhöhen, wie
Wetterbedingungen, Straßenkonditionen, Zustand des Fahrers, Lichtverhältnisse,
Tageszeit und die Verkehrsdichte. \enquote{Machine Learning}-Modelle (ML) sind darauf
ausgelegt, nicht anhand von spezifischen Anweisungen, sondern allein durch das Erkennen
von Mustern und Abhängigkeiten mithilfe komplexer Funktionen \cite{predict}
in Daten Erkenntnisse zu ziehen und Entscheidungen sowie Vorhersagen treffen
zu können \cite{sap}.
\medskip \\
Künstliche neuronale Netze (KNN) sind ein Teilbereich der ML-Modelle. Sie können
Entscheidungen auf ähnlicher Art und Weise treffen, wie das menschliche Gehirn \cite{ibm}.
Das Modell besteht aus einer Vielzahl von Schichten mit unterschiedlich vielen
Knoten. Diese Knoten sind mit anderen Knoten aus den benachbarten Schichten verbunden
und besitzen sogenannte \enquote{weights} und \enquote{bias} \cite{ibm}.
Eingabedaten werden von der \enquote{input layer} (Deutsch: \enquote{Eingangsschicht})
entlang aller Schichten durchgereicht, verarbeitet und schließlich in der
\enquote{output layer} (Deutsch: \enquote{Ausgangsschicht}) aggregiert ausgegeben \cite{ibm}.
\medskip \\
Diese Arbeit wird eine systematische Literaturrecherche (SLR) zur Vorhersage von RTAs
mithilfe KNNs durchführen und abwägen, ob diese mit einem Mehrwert in der Minimierung
von Straßenverkehrsunfällen durch unsichere Straßenarchitektur einhergehen.

\subsection{Aktuelle Forschungslage}
$\ll$ Ausgesuchte Literatur auflisten und den aktuellen Stand erläutern $\gg$.

\section{Methodik}

Diese Arbeit führt eine SLR durch und orientiert sich
dabei an der PRISMA-Leitlinie (Preferred Reporting Items for Systematic Reviews and
Meta-Analyses).  PRISMA ist ein dominanter Reiter bzgl. SLR im medizinischen Sektor,
mit dem Fachkräfte auf dem aktuellen Stand bleiben.  Die Leitlinie wird außerdem als
Basis für zahlreiche medizinische Prozedere und Vorschriften genutzt \cite{prisma}.

\subsection{Die PRISMA-Leitlinie}
PRISMA beinhaltet eine aus 27 Stichpunkten bestehende \enquote{Checkliste} und
ein 4-phasiges Fluss-Diagramm. Mithilfe dieser Elemente kann sowohl Literatur 
systematisch für die Aufnahme in einer SLR auf Eignung geprüft als auch der 
Prozess erleichtert und standardisiert werden.
(Abb. \ref{fig:prisma1}).

\begin{figure}[h]
   \begin{center}
      \includegraphics[width=250px]{Literatur/Bilder/PRISMA-Flow.png}
   \end{center}
   \caption{Ablauf einer SLR nach PRISMA \cite{ex1}}
   \label{fig:prisma1}
\end{figure}

Zunächst werden Forschungsfragen aufgestellt, die im Laufe der Recherche beantwortet
werden sollen. Im Anschluss wird dann die Literatur anhand der ausgewählten 
Such-Strategie ausgewählt. Die gesammelte Literatur wird dann durch Ein- und
Ausschlusskriterien gefiltert und auf Eignung für die SLR durch die Checkliste geprüft.
Schließlich werden aus der eingeschlossenen Literatur qualitative und quantitative
Daten extrahiert, mit anderer Literatur verglichen und in der eigentlichen
Literaturanalyse zusammengetragen.


\subsection{Recherche-Fragen}

$\ll$ Fragen tabellarisch auflisten $\gg$

\subsection{Prozess der Recherche}

Für die Literaturrecherche sind die Datenbanken ACM Digital Library und die IEEE Explore
beansprucht worden. Hierbei wurden die folgenden Suchstrings verwendet:

\begin{tcolorbox}[
      enhanced,
      attach boxed title to top left,
      colback=gray!20,
      colframe=gray,
      colbacktitle=gray,
      title=ACM Digital Library,
      fonttitle=\bfseries\color{black},
      boxed title style={size=small, colframe=gray, sharp corners},
      sharp corners
   ]
   [All: "neural network?"]
   AND [
         [All: "traffic flow"]
         OR [All: "traffic control"]
         OR [All: \dq accident"]
      ]
   AND [E-Publication Date: (01/01/2023 TO 12/31/2024)]
\end{tcolorbox}

\begin{tcolorbox}[
      enhanced,
      attach boxed title to top left,
      colback=gray!20,
      colframe=gray,
      colbacktitle=gray,
      title=IEEE Explore,
      fonttitle=\bfseries\color{black},
      boxed title style={size=small, colframe=gray, sharp corners},
      sharp corners
   ]
   (\dq All Metadata\dq: \dq artificial neural network?")
   AND (\dq All Metadata":"traffic")
   AND (\dq All Metadata":"control\dq\space
   OR \dq All Metadata":\dq accident")
\end{tcolorbox}

Zusätzlich ist die Suche auf den Inhaltstyp \enquote{Research Article}
und die Verfügbarkeit \enquote{Open Access} begrenzt worden, soweit von
den Einstellungen der Datenbank möglich.

\subsection{In- und Exklusionskriterien}

$\ll$ Auflisten und erläutern $\gg$

\section{Ergebnisse}

$\ll$ Erkenntnisse aus den Studien zusammentragen, auswerten und diskutieren $\gg$

\subsection{Limitationen}

$\ll$ Grenzen dieser Arbeit auflisten $\gg$

\section{Diskussion und Fazit}

\subsection{Diskussion}

$\ll$ Forschungsfrage aufgreifen und beantworten $\gg$

\subsection{Fazit}

$\ll$ Zusammenfassen $\gg$

\subsection{Ausblick auf zukünftige Forschung}

$\ll$ Selbsterklärend $\gg$

\printbibliography

\listoffigures
\listoftables

\end{document}